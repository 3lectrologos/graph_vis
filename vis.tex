\documentclass{standalone}
\PassOptionsToPackage{dvipsnames}{xcolor}
\usepackage{tikz}
\usetikzlibrary{decorations}
\usepackage{pgfplotstable}


% Parameters
\def\axisa{\LARGE$a$}
\def\axisb{\LARGE$b$}
\newcommand\timetext[1]{\LARGE$t=#1$}

\def\xstretch{12}
\def\ystretch{5}
\def\minnodesize{8}
\def\nodemul{7}
\def\minwidth{0.2}
\def\widthmulti{6}
\colorlet{locol}{darkgray!30!white}
\colorlet{hicol}{RoyalPurple}

\definecolor{col0}{HTML}{fff7f3}
\definecolor{col1}{HTML}{fde0dd}
\definecolor{col2}{HTML}{fcc5c0}
\definecolor{col3}{HTML}{fa9fb5}
\definecolor{col4}{HTML}{f768a1}
\definecolor{col5}{HTML}{dd3497}
\definecolor{col6}{HTML}{ae017e}
\definecolor{col7}{HTML}{7a0177}
\definecolor{col8}{HTML}{49006a}

\newcommand\col[1]{col#1}

% https://tex.stackexchange.com/questions/14283/stroke-with-variable-thickness
\makeatletter

\pgfkeys{/pgf/decoration/.cd,
	start color/.store in =\startcolor,
	end color/.store in   =\endcolor,
	start width/.store in =\startwidth,
	end width/.store in   =\endwidth
}

\pgfdeclaredecoration{width and color change}{initial}{
	\state{initial}[width=0pt, next state=line, persistent precomputation={%
		\pgfmathdivide{50}{\pgfdecoratedpathlength}%
		\let\increment=\pgfmathresult%
		\def\x{0}%
	}]{}
	\state{line}[width=0.5pt,
	  persistent postcomputation={%
		\pgfmathadd@{\x}{\increment}%
		\let\x=\pgfmathresult%
	}]{%
		\pgfsetlinewidth{\endwidth*(\x/100)+\startwidth*(100-\x)/100}%
		%\pgfsetlinewidth{\pgflinewidth}%
		\pgfsetarrows{-}%
		\pgfpathmoveto{\pgfpointorigin}%
		\pgfpathlineto{\pgfqpoint{0.75pt}{0pt}}%
		\pgfsetstrokecolor{\endcolor!\x!\startcolor}%
		\pgfusepath{stroke}%
	}
	\state{final}{%
		\pgfsetlinewidth{\endwidth}%
		\pgfpathmoveto{\pgfpointorigin}%
        %\pgfpathlineto{\pgfpointdecoratedpathlast\advance\pgf@... by2.0\pgflinewidth}%
		\color{\endcolor!\x!\startcolor}%
		\pgfusepath{stroke}% 
	}
}

\makeatother


\pgfplotstableread{nodes.txt}\ntable
\pgfplotstablegetrowsof{\ntable}
\pgfmathsetmacro{\nrows}{\pgfplotsretval-1}
\pgfplotstableread{edges.txt}\etable
\pgfplotstablegetrowsof{\etable}
\pgfmathsetmacro{\erows}{\pgfplotsretval-1}
\pgfplotstableread{times.txt}\ttable
\pgfplotstablegetrowsof{\ttable}
\pgfmathsetmacro{\trows}{\pgfplotsretval-1}


\begin{document}
\begin{tikzpicture}[
  scale=1,
  nodestyle/.style={circle,draw=darkgray,line width=0.8pt,fill=darkgray,minimum size=0pt,inner sep=0, anchor=center},
  axisstyle/.style={darkgray, line width=0.8pt}
  ]

  \foreach \row in {0,...,\nrows} {
    \pgfplotstablegetelem{\row}{[index]0}\of\ntable
    \xdef\noderow{\pgfplotsretval}
    \pgfplotstablegetelem{\row}{[index]1}\of\ntable
    \xdef\nodecol{\pgfplotsretval}
    \pgfplotstablegetelem{\row}{[index]2}\of\ntable
    \xdef\nodepos{\pgfplotsretval}
    \node [nodestyle] (\noderow/\nodecol) at (\xstretch*\nodepos, -\ystretch*\noderow) {};
  }

    \foreach \row in {0,...,\trows} {
  	\node[minimum size=0,inner sep=0] (s\row) at (-0*\xstretch, -\ystretch*\row) {};
  	\node[minimum size=0,inner sep=0] (e\row) at (1*\xstretch, -\ystretch*\row) {};
  	\draw[axisstyle] (s\row) -- (e\row);
  	\node[anchor=south] (a\row) at (-0*\xstretch, -0.8-\ystretch*\row) {\axisa};
  	\node[anchor=south] (b\row) at (1*\xstretch, -0.8-\ystretch*\row) {\axisb};
  	\pgfplotstablegetelem{\row}{[index]0}\of\ttable
  	\xdef\time{\pgfplotsretval}
  	\node[anchor=east] (foo\row) at (-0.08*\xstretch, -\ystretch*\row) {\timetext{\time}};
  }

  \foreach \row in {0,...,\erows} {
	\pgfplotstablegetelem{\row}{[index]0}\of\etable
	\xdef\srow{\pgfplotsretval}
	\pgfplotstablegetelem{\row}{[index]1}\of\etable
	\xdef\scol{\pgfplotsretval}
	\pgfplotstablegetelem{\row}{[index]2}\of\etable
	\xdef\trow{\pgfplotsretval}
	\pgfplotstablegetelem{\row}{[index]3}\of\etable
	\xdef\tcol{\pgfplotsretval}
	% Colors
	\pgfplotstablegetelem{\row}{[index]4}\of\etable
	\xdef\sclr{\pgfplotsretval}
	\colorlet{cstart}{\col{\sclr}}
	%\colorlet{cstart}{hicol!\sclr!locol}
	\pgfplotstablegetelem{\row}{[index]5}\of\etable
	\xdef\tclr{\pgfplotsretval}
	\colorlet{cend}{\col{\tclr}}
	%\colorlet{cend}{hicol!\tclr!locol}
	% Widths
	\pgfplotstablegetelem{\row}{[index]6}\of\etable
	\xdef\swidth{\minwidth + \widthmulti*\pgfplotsretval}
	\pgfplotstablegetelem{\row}{[index]7}\of\etable
	\xdef\twidth{\minwidth + \widthmulti*\pgfplotsretval}
	%\draw [decoration={width and color change, start color=darkgray!80!white, end color=darkgray!80!white, start width=3*\swidth, end width=3*\twidth}, decorate] (\srow/\scol.center) to [out=-90,in=90,looseness=0.8] (\trow/\tcol.center);
	\draw [decoration={width and color change, start color=cstart, end color=cend, start width=\swidth, end width=\twidth}, decorate] (\srow/\scol.center) to [out=-90,in=90,looseness=0.8] (\trow/\tcol.center);
  }
  
  \foreach \row in {0,...,\nrows} {
  	\pgfplotstablegetelem{\row}{[index]0}\of\ntable
  	\xdef\noderow{\pgfplotsretval}
  	\pgfplotstablegetelem{\row}{[index]1}\of\ntable
  	\xdef\nodecol{\pgfplotsretval}
  	\pgfplotstablegetelem{\row}{[index]2}\of\ntable
  	\xdef\nodepos{\pgfplotsretval}
  	\pgfplotstablegetelem{\row}{[index]3}\of\ntable
  	\xdef\nodecolor{\pgfplotsretval}
  	\colorlet{nodecol}{\col{\nodecolor}}
  	%\colorlet{nodecol}{hicol!\nodecolor!locol}
  	\pgfplotstablegetelem{\row}{[index]4}\of\ntable
  	\xdef\nodesize{\minnodesize+\nodemul*\pgfplotsretval}
  	\node [nodestyle,fill=nodecol,minimum size=\nodesize] (\noderow/\nodecol) at (\xstretch*\nodepos, -\ystretch*\noderow) {};
  }


\end{tikzpicture}
\end{document}